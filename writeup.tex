\documentclass{article}
\usepackage{sagetex}
\usepackage{float}
\usepackage{amsmath}
\usepackage{amsfonts}
\usepackage{listings}
\usepackage[margin=1in]{geometry}

\title{Numerical solution of the time-independent Schr\"{o}dinger Equation}
\author{Feynman Liang}
\date{\today}
\begin{document}

\maketitle

\section{Introduction}
Solutions to the Schr\"{o}dinger equation can oftentimes be approximated using numerical
methods. This is especially convenient for dealing with potentials where analytical 
solutions may not exist or be very difficult to find. Here, we will investigate numerical 
solutions to a 1D finite quantum well with potential:

\begin{equation}\label{FiniteBoxPotential}
    \widetilde{V}(u)=
    \begin{cases} 0 & \text{\(\lvert u\rvert\) $\le 0.5$}
    \\
    1 & \text{\(\lvert u\rvert\) $> 0.5$}
\end{cases}
\end{equation}

Rewriting the Schr\"{o}dinger equation in a dimensionless form, we obtain:

\begin{equation}\label{FiniteBoxSchrodinger}
    \frac{d^{2}\psi(u)}{du^{2}} = -\beta\left[\epsilon-\widetilde{V}(u)\right]\psi(u)
\end{equation}

where:

\begin{equation*}\label{DimensionlessSubstitutions}
    u = \frac{x}{a},\;\;
    \epsilon = \frac{E}{V_{0}},\;\;
    \beta = \frac{2ma^{2}V_0}{\hbar^2}
\end{equation*}

Since we expect energies below $V_0$ to be bound, there is a boundary condition
$\psi(u) = 0 (x \rightarrow \infty)$. This boundary condition problem can be
made suitable for numerical computation by employing the shooting method, which
reduces the problem into an initial value problem (Note that unlike traditional
IVPs, here we are choosing values for $\epsilon$ rather than $\psi(0)$ and
dealing with asymptotic boundary conditions). Using the shooting method, we
will guess and successively adjust the energy term $\epsilon$ until $\psi(u)$
satisfies the desired asymptotic behavior. Values of $\epsilon$ which satisfy
the boundary conditions are allowed energy eigenvalues for the
$\widetilde{V}(u)$ potential
in \eqref{FiniteBoxPotential}.

\section{Numerical Approximation}

The accompanying code was run using Sage 5.5 and Python 3.3.0.  The Sage\TeX\,
package is used to include excecution results in-line.

Two global variables we will use are $\beta$ (=64, see ER Appendix G) and
the potential function $\widetilde{V}(u)$. These are specified at the top of
the program:

\begin{sageblock}
    V_potential = lambda u: 1 if (abs(u) > 1/2) else 0
    beta = 64
\end{sageblock}

In order to numerically approximate the limit behavior of $\psi(u)$, we define
a function \texttt{shooting\_solver\_1d\_finite} to numerically integrate the dimensionless
Schr\"{o}dinger equation \eqref{FiniteBoxSchrodinger}. Our implementation uses the
Forward Euler method, which computes the quadrature using iterative first order
Taylor approximations:

\begin{align}\label{IterationDef}
    \phi(u_{i+1}) &= \phi(u_i) - \Delta u \cdot \beta \left[\epsilon - \widetilde{V}(u_i)\right]\psi(u_i)\\
    \psi(u_{i+1}) &= \psi(u_i) + \Delta u \cdot \phi(u_i)
\end{align}

The function \texttt{shooting\_solver\_1d\_finite} takes initial values \texttt{psi0} and
\texttt{dpsi0} ($\psi$ and $\frac{d\psi}{du}$ at $x_0 = 0$), a guess for
\texttt{epsilon}, an iteration step size \texttt{delta\_u}. A list of tuples
$(u, \psi(u))$ is returned for the range $[$0, \texttt{uf}$]$.

\begin{sageblock}
def shooting_solver_1d_finite(psi0, dpsi0, uf, epsilon, delta_u):
    num_steps = (uf - 0) / delta_u
    data = [(0, psi0, dpsi0)] # initialize data array
    for i in range(num_steps): # perform forward euler
        u_old = data[i][0]
        psi_old = data[i][1]
        dpsi_old = data[i][2]
        u = u_old + delta_u
        # Taylor approximations given by (3) and (4)
        dpsi = dpsi_old - delta_u * beta \
               * (epsilon - V_potential(u_old)) * psi_old
        psi = psi_old + delta_u * dpsi_old
        data.append((u, psi, dpsi))
    return map(lambda x: (x[0], x[1]), data) # return list of (u, psi) tuples
\end{sageblock}

Because the potential \eqref{FiniteBoxPotential} is symmetric
($\widetilde{V}(u) = \widetilde{V}(-u)$), the energy eigenfunctions have a
definite parity. Thus, the eigenfunctions exhibit symmetry about the origin and 
a simulation between $(0, \texttt{uf})$ will also yield the result between
$(-\texttt(uf), 0)$ by simply mirroring across the y-axis and multiplying
$\psi(u)$ by $-1$ if the eigenfunction is of odd parity.

The \texttt{plot\_data\_finite\_well} function takes a list \texttt{data} of
(u, $\psi(u)$) 2-tuples, the wavenumber \texttt{n} to label the y-axis with,
a \texttt{title} string, and uses the matplotlib library to generates a plot of
\texttt{data} with dashed lines indicating the boundaries of the finite well
potential \eqref{FiniteBoxPotential}. The \texttt{chained} argument is an implementation
detail with no effect on the solution and can safely be ignored.

\emph{\textbf{Note}: This plotting routine assumes the data approximates an
eigenfunction of definite parity.}
\begin{sageblock}
import matplotlib.pyplot as plt
def plot_finite_well(data, n, title, chained=False):
    if not chained: plt.clf() # only clear if not part of chained call
    # mirror data across origin, assumes data has a definite parity
    (u, psi0) = data[0]
    if psi0 == 0: # node at 0, odd
        data = map(lambda x: (-x[0], -x[1]), data)[::-1] + data
    else: # definite parity => anti-node at 0, even
        data = map(lambda x: (-x[0], x[1]), data)[::-1] + data

    u, psi = [[x[i] for x in data] for i in (0,1)]
    plt.plot(u, psi, label="Euler Approximation")
    plt.title(title)
    plt.xlabel('$u$')
    plt.ylabel("$\\psi_{%s}(u)$" % n)
    plt.axvline(x=0, color='black')
    plt.axvline(x=0.5, linestyle='dashed', color='black')
    plt.axvline(x=-0.5, linestyle='dashed', color='black')
    plt.grid(True)
    plt.legend()
    plt.save = plt.savefig
    return plt 
\end{sageblock}

\section{Simulation and Results}
\subsection{Ground State of Finite Quantum Well} Because the symmetric
potential implies energy eigenfunctions of definite parity and the ground state
eigenfunction is odd, $\frac{d\psi(u)}{du}$ must necessarily be 0 at $x_0 = 0$. 

Rather than worry about the normalization conditions for $\psi(u)$
$\left(\int_{-\infty}^{\infty}\psi^\dag\psi du = 1\right)$, we recognize that
the normalized eigenfunction is simply the unnormalized eigenfunction
multiplied by a constant normalization factor. We can compute the normalization
factor given the unormalized eigenfunction, so we make the assumption that
$\psi = 1.0$ at $x_0 = 0$, recognizing that we are working with the unnormalized
eigenfunction. This does not affect our numerical results for $\epsilon$
because looking at \eqref{FiniteBoxSchrodinger} shows that the normalization
constants cancel. 

In order to plot our solutions, our simulation will only find eigenfunctions which are
real functions. This is not a problem because we have the ability to choose the phase of 
solutions to the TISE \eqref{FiniteBoxSchrodinger} to make $\psi(u) \in \mathbb{R}$ (see P24 - Computation Project Handout).

Using these initial conditions, the "shooting" method to determine $\epsilon$ yields:

\begin{figure}[H]
\begin{sageblock}
plot_finite_well(shooting_solver_1d_finite(1, 0, 1.5, .01, 0.0001), \
                 n=1, title="Guess $\epsilon=0.01$")
\end{sageblock}
\vspace{-.1in}
\centering
\sageplot[width=300px, ]{plot_finite_well(shooting_solver_1d_finite(1, 0, 1.5, .01, 0.0001), n=1, title="Guess $\epsilon=0.01$")}
\end{figure}
\vspace{-.2in}
\begin{figure}[H]
\begin{sageblock}
plot_finite_well(shooting_solver_1d_finite(1, 0, 1.5, .09, 0.0001), \
                 n=1, title="Guess $\epsilon=0.09$")
\end{sageblock}
\vspace{-.1in}
\centering
\sageplot[width=300px]{plot_finite_well(shooting_solver_1d_finite(1, 0, 1.5, .09, 0.0001), n=1, title="Guess $\epsilon=0.09$")}
\end{figure}
\vspace{-.2in}
\begin{figure}[H]
\begin{sageblock}
plot_finite_well(shooting_solver_1d_finite(1, 0, 1.5, .095, 0.0001), \
                 n=1, title="Guess $\epsilon=0.095$")
\end{sageblock}
\vspace{-.1in}
\centering
\sageplot[width=300px]{plot_finite_well(shooting_solver_1d_finite(1, 0, 1.5, .095, 0.0001), n=1, title="Guess $\epsilon=0.095$")}
\end{figure}
\vspace{-.2in}
\begin{figure}[H]
\begin{sageblock}
plot_finite_well(shooting_solver_1d_finite(1, 0, 1.5, .0975, 0.0001), \
                 n=1, title="Guess $\epsilon=0.0975$")
\end{sageblock}
\vspace{-.1in}
\centering
\sageplot[width=300px]{plot_finite_well(shooting_solver_1d_finite(1, 0, 1.5, .0975, 0.0001), n=1, title="Guess $\epsilon=0.0975$")}
\end{figure}
\vspace{-.2in}
\begin{figure}[H]
\begin{sageblock}
plot_finite_well(shooting_solver_1d_finite(1, 0, 1.5, .098, 0.0001), \
                 n=1, title="Guess $\epsilon=0.098$")
\end{sageblock}
\vspace{-.1in}
\centering
\sageplot[width=300px]{plot_finite_well(shooting_solver_1d_finite(1, 0, 1.5, .098, 0.0001), n=1, title="Guess $\epsilon=0.098$")}
\end{figure}
\vspace{-.2in}
\begin{figure}[H]
\begin{sageblock}
plot_finite_well(shooting_solver_1d_finite(1, 0, 1.5, .097993, 0.0001), \
                 n=1, title="Ground Energy Eigenfunction, $\epsilon=0.097993$")
\end{sageblock}
\vspace{-.1in}
\centering
\sageplot[width=300px]{plot_finite_well(shooting_solver_1d_finite(1, 0, 1.5, .097993, 0.0001), n=1, title="Ground Energy Eigenfunction, $\epsilon=0.097993$")}
\end{figure}

This shows that a valid solution is found at $\epsilon = 0.097993$, implying
that an energy eigenstate exists with eigenvalue $0.097993V_0$, where $V_0$ is
the height of the finite well potential. We see that there is a single node in
this wavefunction, indicating that this eigenstate is the ground state.

This solution agrees quite well with the analytic solution $\epsilon = 0.0980$
(ER Appendix H).

\subsection{Excited States of Finite Quantum Well}

Bound states will exist so long as $E < V_0 \equiv \epsilon < 1$ so we can
search for additional energy eigenstates by guessing larger $\epsilon$. The
first excited state will have a node at the origin ($\psi(0) = 0$). The
linearity property of the Schr\"{o}dinger equation \eqref{FiniteBoxSchrodinger} as
well as our focus on the unnormalized eigenfunction allows us to choose any
arbitrary value for $\frac{d\psi}{du}$ (we will use $\frac{d\psi}{du}=1$).

Running the simulation using these initial conditions (multiple trials omitted)
an energy eigenvalue to $V_0$ proportionality constant of $\epsilon = 0.382605$
for $\psi_2$:

\begin{figure}[H]
\begin{sageblock}
plot_finite_well(shooting_solver_1d_finite(0, 1, 2, .382605, 0.0001), \
                 n=2, title="1st Excited Energy Eigenfunction, $\epsilon=0.382605$")
\end{sageblock}
\vspace{-.1in}
\centering
\sageplot[width=300px]{plot_finite_well(shooting_solver_1d_finite(0, 1, 2, .382605, 0.0001), n=2, title="1st Excited Energy Eigenfunction, $\epsilon=0.382605$")}
\end{figure}

For $\psi_3$, $\epsilon = 0.80767$:

\begin{figure}[H]
\begin{sageblock}
plot_finite_well(shooting_solver_1d_finite(-1, 0, 2, .80767, 0.0001), \
                      n=3, title="2nd Excited Energy Eigenfunction, $\epsilon=0.80767$")
\end{sageblock}
\vspace{-.1in}
\centering
\sageplot[width=300px]{plot_finite_well(shooting_solver_1d_finite(1, 0, 2, .80767, 0.0001), n=3, title="2nd Excited Energy Eigenfunction, $\epsilon=0.80767$")}
\end{figure}

\subsubsection{Other Numerical Methods} 
The shooting method provides a convenient way to numerically solve the
finite-well problem by converting the boundary value problem to an initial
value problem. However, this is not a traditional boundary value problem.
Rather than solving for the values of $\psi(u)$ given the boundary constraints,
we are solving for the $\epsilon$ value in the differential equation which give
rise to solutions satisfying the boundary conditions. Additionally, we are dealing
with asymptotic boundary conditions.

Nonetheless, we can reformulate techniques such as finite difference methods to
accompany the additional variable $\epsilon$. The asymptotic boundary condition
can be mitigated by choosing the boundary to be at a large value of $u$. By
treating $\epsilon$ as an unknown, the multiplication of $\epsilon$ with
$\psi(u)$ in equation \eqref{FiniteBoxSchrodinger} will result in a system of
non-linear equations.  This system of non-linear equations can then be solved
using Newton's method, yielding both the eigenfunction $\psi(u)$ as well as
$\epsilon$. 

In addition, we can also improve on the method for refining our guess for $\epsilon$. 
Rather than manually update our guess, a binary search procedure could be used to refine
subsequent guesses for $\epsilon$ until a desired error threshold is reached.

\subsection{"Scattering" States of Finite Quantum Well}\label{sec:ScatterStates}

States where ($E > V_0 \rightarrow \epsilon > 1$) are where the quantum well is
no longer able to bind the particle. For these unbound energy states, the
allowable energy levels are no longer discrete. One example of an acceptable
unbound solution to equation \eqref{FiniteBoxSchrodinger} is when $\epsilon = 3$:

\begin{figure}[H]
\begin{sageblock}
plot_finite_well(shooting_solver_1d_finite(-1, 0, 5, 3, 0.0001), \
                 n="unbound", title="Unbound Energy State, \epsilon=3")
\end{sageblock}
\vspace{-.1in}
\centering
\sageplot[width=300px]{plot_finite_well(shooting_solver_1d_finite(-1, 0, 5, 3, 0.0001), n="unbound", title="Unbound Energy State, \epsilon=3")}
\end{figure}

Any $\epsilon > 1$ is a valid energy which will tend towards a sinusoidal steady
state in the limit $u \rightarrow \infty$. As energy is increased, so does the
frequency of $\psi$. In the limit $E \rightarrow \infty$, the probability
density approaches the clasically expected uniform probability density. 

\subsubsection{Step size error}\label{StepSizeError}
This result, however, is quite sensitive to the step size used for numerical
integration. Forward Euler is a first order solver which can be numerically
unstable and introducing additional energy. For example, the results from
\ref{sec:ScatterStates} no longer results in sinusoidal limit behavior when $\Delta
U$ is changed from $0.0001$ to $0.01$.  Rather, the Forward Euler approximation
adds enough energy to result in infinite blowup:

\begin{figure}[H]
\begin{sageblock}
plot_finite_well(shooting_solver_1d_finite(-1, 0, 5, 3, 0.01), \
                      n="unbound", title="Energy gain due to low step size, \
                                          $\epsilon=3$")
\end{sageblock}
\vspace{-.1in}
\centering
\sageplot[width=300px]{plot_finite_well(shooting_solver_1d_finite(-1, 0, 5, 3, 0.01), n="unbound", title="Energy gain due to low step size, $\epsilon=3$")}
\end{figure}

To address the numerical stability issues of Forward-Euler as well as increase
rate of convergance of error to $0$ as $\Delta U \rightarrow \infty$, implicit
and symplectic solvers of higher order can be used to carry out numerical
integration.

\section{Infinite Quantum Well}\label{sec:InfWell}

The simulation code can be easily adjusted for different potentials. For
example, the infinite quantum well can be simulated by modifying the definition
of the potential:

\begin{sageblock}
# 999 to approximate infinite potential
V_potential = lambda u: 999 if (abs(u) > 1/2) else 0
\end{sageblock}

Next, by interpreting $V_0 = 1$ eV and using the boundary condition
$\psi(\text{\(\lvert u\rvert\)}=\frac{1}{2}) = 0$, the energy eigenstates can be found.

Since we have an analytical solution for an infinite square well (ER Eq. 6-79,
Eq. 6-80):

\begin{equation}
    \psi_n(x) = 
    \begin{cases}
        B_n \cos k_n x &\text{where } k_n = \frac{n\pi}{a},\; n = 1,3,5,\ldots\\
        A_n \sin k_n x &\text{where } k_n = \frac{n\pi}{a},\; n = 2,4,6,\ldots\\
    \end{cases}
\end{equation}

We can visualize how well our simulation approximates analytical results by modifying our
plotting function to also plot the analytical solutions. The constant factors $A_n$ and $B_n$
are chosen to be consistent with initial conditions:

\begin{align}
    \psi(0) &= B_n \\
    \frac{d\psi}{du}\bigg|_{u=0} &= \frac{d}{dx}A_n \sin k_n a\cdot0 \nonumber \\
                        &= A_n k_n \cos k_n a\cdot0 \nonumber \\
                        &= A_n k_n
\end{align}

Solving these using the initial conditions $\psi(0)=1$ and
$\frac{d\psi}{du}\big|_{u=0}$ yields $A_n = \frac{1}{k_n}$ and $B_n =
\psi(0)$, which we subsequently use to define a \texttt{plot\_inf\_well} method
which takes a quantum number \texttt{n}, a \texttt{data} list of simulation
results, and a \texttt{title} string.  This function will first plot the
analytical solution for $\psi_n(u)$ in an $\infty$-square potential and
subsequently passes control to \texttt{plot\_finite\_well} to plot the
simulation results.

\begin{sageblock}
from numpy import arange
def plot_inf_well(n, uf, data, title):
    plt.clf()
    u = arange(-float(uf),float(uf),2*float(uf)/100)
    if n % 2 != 0: # 
        psi = [1 * cos(n*i*pi) for i in u]
    else:
        psi = [(1 / (n * pi)) * sin(n*i*pi) for i in u]
    plt.plot(u, psi, label="Analytical Solution", marker='+', linestyle='None', color='Green')
    plt.save = plt.savefig
    return plot_finite_well(data, n, title, chained=True)
\end{sageblock}

This results in (multiple trials omitted):

\begin{figure}[H]
\begin{sageblock}
plot_inf_well(1, 0.5, shooting_solver_1d_finite(1, 0, 0.5, .1542, 0.0001), \
              title="Ground Energy Eigenfunction, $\infty$-quantum well, $\epsilon=0.1542$")
\end{sageblock}
\vspace{-.1in}
\centering
\sageplot[width=300px]{plot_inf_well(1, 0.5, shooting_solver_1d_finite(1, 0, 0.5, .1542, 0.0001), title="Ground Energy Eigenfunction, $\infty$-quantum well, $\epsilon=0.1542$")}
\end{figure}

\begin{figure}[H]
\begin{sageblock}
plot_inf_well(2, 0.5, shooting_solver_1d_finite(0, 1, 0.5, 0.6179, 0.0001), \
              title="First Exctited Energy Eigenfunction, $\infty$-quantum well, \
                     $\epsilon=0.6179$")
\end{sageblock}
\vspace{-.1in}
\centering
\sageplot[width=300px]{plot_inf_well(2, 0.5, shooting_solver_1d_finite(0, 1, 0.5, 0.6179, 0.0001), title="First Exctited Energy Eigenfunction, $\infty$-quantum well, \epsilon=0.6179")
}
\end{figure}

We see that our simulated eigenfunctions follow the analytical solutions very
closely. The results indicate that the ground and first excited energy
eigenstates for this $\infty$-quantum well have energy eigenvalues of 0.1542 eV
and 0.6179 eV.  Compared to the energy eigenvalues obtained analytically (ER Eq.
6-81):

\begin{align}
    m &= \frac{\beta\hbar^2}{2a^2V_0} \:(\text{from } \eqref{DimensionlessSubstitutions})\nonumber \\
    E_n &= \frac{\pi^2 \hbar^2 n^2}{2ma^2} = \frac{\pi^2n^2V_0}{\beta} \nonumber \\
    E_1 &= \frac{\pi^2}{64} \approx 0.1542 \\
    E_2 &= \frac{4\pi^2}{64} \approx 0.6168 
\end{align}

we see that our simulation yields an answer accurate to the hundredth of an eV.

\section{Quantum Harmonic Oscillator}
The numerical methods developed so far can be extended to other non-trivial potentials. For a 
quantum harmonic oscillator, the potential is defined by $V(x) =
\frac{1}{2}Cx^2$ and has a resonance frequency of $\omega =
esqrt{\frac{C}{m}}$.  We can reformulate the time-independent Schr\"{o}dinger
equation for the quantum harmonic oscillator into a dimensionless form (P24 -
PS1 Problem 3):

\begin{equation}
    \frac{d^2\psi}{du^2} + (\epsilon - u^2)\psi = 0
\end{equation}

where:

\begin{equation}
    \epsilon = \frac{2E}{\hbar\omega},\;\;
    u = \frac{x}{x_0},\;\;
    x_0 = \sqrt{\frac{\hbar}{m\omega}}\;\;
\end{equation}

While section \ref{sec:InfWell} illustrated that the potential can be modified by simply changing
one line, because we have reparameterized the TISE to a dimensionless quantum harmonic oscillator
case, we will need to define a new \texttt{shooting\_solver\_qho} to account for these new changes. 
Additionally, we will define a modified \texttt{plot\_qho} which does not plot
dashed lines at $u=\pm0.5$ as those are meaningless in this context:

\begin{sageblock}
hbar = 6.58211928*10**-16 # eV s
V_potential = lambda u: C * (u * x0)**2 / 2
C = 1
m = 1
omega = sqrt(C / m)
x0 = sqrt(hbar / (m * omega))

def shooting_solver_qho(psi0, dpsi0, uf, epsilon, delta_u):
    num_steps = (uf - 0) / delta_u
    data = [(0, psi0, dpsi0)] # initialize data array
    for i in range(num_steps): # perform forward euler
        u_old = data[i][0]
        psi_old = data[i][1]
        dpsi_old = data[i][2]
        u = u_old + delta_u
        # Taylor approximations using (10)
        dpsi = dpsi_old - delta_u * (epsilon - u**2) * psi_old
        psi = psi_old + delta_u * dpsi_old
        data.append((u, psi, dpsi))
    return map(lambda x: (x[0], x[1]), data) # return list of (u, psi) tuples

def plot_qho(data, n, title, chained=False):
    if not chained: plt.clf() # only clear if not part of chained call
    # mirror data across origin, assumes data has a definite parity
    (u, psi0) = data[0]
    if psi0 == 0: # node at 0, odd
        data = map(lambda x: (-x[0], -x[1]), data)[::-1] + data
    else: # definite parity => anti-node at 0, even
        data = map(lambda x: (-x[0], x[1]), data)[::-1] + data

    u, psi = [[x[i] for x in data] for i in (0,1)]
    plt.plot(u, psi, label="Euler Approximation")
    plt.title(title)
    plt.xlabel('$u$')
    plt.ylabel("$\\psi_{%s}(u)$" % n)
    plt.axvline(x=0, color='black')
    plt.grid(True)
    plt.legend()
    plt.save = plt.savefig
    return plt 
\end{sageblock}

Again, we use the symmetry of the potential and only calculate for $u \ge 0$, appropriately
mirroring the results inside \texttt{plot\_qho}. Also, the definite parity of
eigenfunctions due to the symmetric potential and the linearity of eigenstates
again allow us to arbitrarily choose initial conditions of $\psi(0)=1$ for
even eigenfunctions and $\frac{d\psi}{du}\big|_{u=0}=1$ for odd
eigenfunctions. We can now repeatedly guess and refine an estimate for $\epsilon$ 
which causes $\psi$ to obey the asymptotic bound $\psi(u) = 0$ as $u \rightarrow \infty$ (multiple
trials omitted):

\begin{figure}[H]
\begin{sageblock}
plot_qho(shooting_solver_qho(1, 0, 3.8, 1.00005, 0.0001), \
              n=1, title="QHO Ground State, $\epsilon=1.00005$")
\end{sageblock}
\vspace{-.1in}
\centering
\sageplot[width=300px]{plot_qho(shooting_solver_qho(1, 0, 3.9, 1.00005, 0.0001), n=1, title="QHO Ground State, $\epsilon=1.00005$")}
\end{figure}

The first excited state can be found by requiring a node at $u = 0$, which we
do by setting the initial condition $\psi(0)=0$. Since the first excited state 
for a QHO is odd, we agreed during the above discussion of initial conditions to
use $\frac{d\psi}{du}\big|_{u=0}=1$:

\begin{figure}[H]
\begin{sageblock}
plot_qho(shooting_solver_qho(0, 1, 3.9, 3, 0.0001), \
              n=2, title="QHO First Excited State, $\epsilon=3.9$")
\end{sageblock}
\vspace{-.1in}
\centering
\sageplot[width=300px]{plot_qho(shooting_solver_qho(0, 1, 3.9, 3, 0.0001), n=2, title="QHO First Excited, $\epsilon=3.9$")}
\end{figure}

Our results estimate the dimensionless energy parameter
$\epsilon=\frac{2E}{\hbar\omega}$ to be $\epsilon_1=1.00005$ in the ground
state and $\epsilon_2=3.9$ in the first excited state. This agrees with
analytical results (P24 - PS3 - Problem 3c), which require $\epsilon_n = 1, 3,
5, \ldots$. 

\section{Conclusions}

From the above studies, it is clear that the shooting method coupled with
numerical integration using Forward Euler can produce results which closely
mirror analytic solutions. The numerical approximation functions presented not
only solve the 1D finite well with results very close to analytical solutions,
but are also modular and easily modifiable. This was illustrated by the
relatively small amount of effort required to reconfigure our program to solve
the infinite well potential and quantum harmonic oscillator potential. In both
of these latter examples, numerical techniques achieved results in close
agreement with analytical solutions.

By decreasing the step size used for numerical integration, the error
introduced by Forward Euler should converge towards zero, yielding a flexible
method to numerically integrate open-form differential equations. Coupled with
the plotting routines presented, this makes for a convenient method to
approximating energy eigenvalues and eigenstates for arbitrary 1D potentials.

Nonetheless, this method is far from perfect. We saw in section
\ref{StepSizeError} that Forward Euler violates the conservation of energy and
can result in erroneous asymptotic behavior. Decreasing the step size is one
solution to this problem, but doing so also increases the amount of work done
by the computer. Other methods of increasing the rate of error convergence to 0
as $\Delta U \rightarrow \infty$ include using a higher-order or
implicit/symplectic solver for the numerical integration process. To improve
the precision of $\epsilon$, the repetitive guess-refine feedback loop
necessary for the shooting method can be itself made a callable function with a
configurable termination condition.



\end{document}
