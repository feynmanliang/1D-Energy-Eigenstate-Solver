\documentclass{article}
\usepackage{sagetex}
\usepackage{float}
\usepackage{amsmath}
\usepackage{amsfonts}
\usepackage{listings}
\usepackage[margin=1in]{geometry}

\title{Numerical solution of the time-independent Schr\"{o}dinger Equation}
\author{Feynman Liang}
\date{\today}
\begin{document}

\maketitle

\section{Introduction}
Solutions to the Schr\"{o}dinger equation can oftentimes be approximated using numerical
methods. This is especially convenient for dealing with potentials where analytical 
solutions may not exist or be very difficult to find. Here, we will investigate numerical 
solutions to a 1D finite quantum well with potential:

\begin{equation}\label{FiniteBoxPotential}
    \widetilde{V}(u)=
    \begin{cases} 0 & \text{\(\lvert u\rvert\) $\le 0.5$}
    \\
    1 & \text{\(\lvert u\rvert\) $> 0.5$}
\end{cases}
\end{equation}

Rewriting the Schr\"{o}dinger equation in a dimensionless form, we obtain:

\begin{equation}\label{FiniteBoxSchrodinger}
    \frac{d^{2}\psi(u)}{du^{2}} = -\beta\left[\epsilon-\widetilde{V}(u)\right]\psi(u)
\end{equation}

where:

\begin{equation*}\label{DimensionlessSubstitutions}
    u = \frac{x}{a},\;\;
    \epsilon = \frac{E}{V_{0}},\;\;
    \beta = \frac{2ma^{2}V_0}{\hbar^2}
\end{equation*}

Since we expect energies below $V_0$ to be bound, there are boundary conditions
$\psi(u) = 0$ and $\frac{d\psi(u)}{du} = 0$ for $u \rightarrow \pm \infty$.
This boundary condition problem can be made suitable for numerical computation
by employing the shooting method, which reduces the problem into an initial
value problem (Note that unlike traditional IVPs, here we are choosing an initial
value for $\epsilon$ rather than $\psi(0)$ and dealing with asymptotic rather than
explicit boundary conditions). Using the shooting method and visual inspection,
we will guess and successively adjust the energy term $\epsilon$ until
$\psi(u)$ satisfies the desired asymptotic behavior within the range examined.
The $\epsilon$ values resulting from this method are allowed energy eigenvalues
for the $\widetilde{V}(u)$ potential in \eqref{FiniteBoxPotential}.

\section{Theory and Methods}

The code was run using Sage 5.5 and Python 3.3.0.  The Sage\TeX\, package is
used to include evaluation results in-line.

Two global variables we will use are $\beta$ (set to 64, see assignment
specifications 6.1) and the potential function $\widetilde{V}(u)$. These are
specified at the top of the program:

\begin{sageblock}
    beta = 64
    V_potential = lambda u: 1 if (abs(u) > 1/2) else 0
\end{sageblock}

In order to numerically approximate the limit behavior of $\psi(u)$, we define
a function \texttt{shooting\_solver\_1d\_finite} to numerically integrate the dimensionless
Schr\"{o}dinger equation \eqref{FiniteBoxSchrodinger}. Our implementation uses the
Forward Euler method, which computes the quadrature using successive first order
Taylor approximations:

\begin{align}\label{IterationDef}
    \phi(u_{i+1}) &= \phi(u_i) - \Delta u \cdot \beta \left[\epsilon - \widetilde{V}(u_i)\right]\psi(u_i)\\
    \psi(u_{i+1}) &= \psi(u_i) + \Delta u \cdot \phi(u_i)
\end{align}

The function \texttt{shooting\_solver\_1d\_finite} takes initial values
\texttt{psi0} and \texttt{dpsi0} ($\psi$ and $\frac{d\psi}{du}$ at $x_0 = 0 =
u$), a guess for \texttt{epsilon}, and a step size \texttt{delta\_u}. A list of
2-tuples $(u, \psi(u))$ is returned for $u \in [$0, \texttt{uf}$]$.

\begin{sageblock}
def shooting_solver_1d_finite(psi0, dpsi0, uf, epsilon, delta_u):
    num_steps = (uf - 0) / delta_u
    data = [(0, psi0, dpsi0)] # initialize data array
    for i in range(num_steps): # perform forward euler
        u_old = data[i][0]
        psi_old = data[i][1]
        dpsi_old = data[i][2]
        u = u_old + delta_u
        # Taylor approximations given by (3) and (4)
        dpsi = dpsi_old - delta_u * beta \
               * (epsilon - V_potential(u_old)) * psi_old
        psi = psi_old + delta_u * dpsi_old
        data.append((u, psi, dpsi))
    return map(lambda x: (x[0], x[1]), data) # return list of (u, psi) tuples
\end{sageblock}

Because the potential \eqref{FiniteBoxPotential} is symmetric about $u=0$
($\widetilde{V}(u) = \widetilde{V}(-u)$), the energy eigenfunctions have a
definite parity. Thus, they exhibit symmetry about the origin and a simulation
result between $(0, \texttt{uf})$ will also reveal the result between
$(-\texttt(uf), 0)$ by simply mirroring across the y-axis and multiplying
$\psi(u)$ by $-1$ if the eigenfunction is of odd parity.

The \texttt{plot\_data\_finite\_well} function takes a list \texttt{data} of
(u, $\psi(u)$) 2-tuples, the wavenumber \texttt{n} used for subscripting the
y-axis label, a \texttt{title} string, and uses the matplotlib plotting library
to generates a plot of \texttt{data} with dashed lines illustrating the
boundaries of the finite well potential \eqref{FiniteBoxPotential}. This
function takes care of mirroring the data from the simulation across $u=0$
appropriately. The \texttt{chained} argument is an implementation detail with
no effect on the solution (it prevents the plotting area from being cleared,
allowing us to chain the infinite square well plotting function with
\texttt{plot\_finite\_well} and reduce redundant code):

\emph{\textbf{Note}: This plotting routine assumes the eigenfunction is odd if
$\psi(0)=0$ and even otherwise.}

\begin{sageblock}
import matplotlib.pyplot as plt
def plot_finite_well(data, n, title, chained=False):
    if not chained: plt.clf() # only clear if not part of chained call
    # mirror data across origin, assumes data has a definite parity
    (u, psi0) = data[0]
    if psi0 == 0: # node at 0, odd
        data = map(lambda x: (-x[0], -x[1]), data)[::-1] + data
    else: # definite parity => anti-node at 0, even
        data = map(lambda x: (-x[0], x[1]), data)[::-1] + data

    u, psi = [[x[i] for x in data] for i in (0,1)]
    plt.plot(u, psi, label="Euler Approximation")
    plt.title(title)
    plt.xlabel('$u$')
    plt.ylabel("$\\psi_{%s}(u)$" % n)
    plt.axvline(x=0, color='black')
    plt.axvline(x=0.5, linestyle='dashed', color='black')
    plt.axvline(x=-0.5, linestyle='dashed', color='black')
    plt.grid(True)
    plt.legend()
    plt.save = plt.savefig # hack for proper SageTeX behavior
    return plt 
\end{sageblock}

\section{Results}
\subsection{Ground State of Finite Quantum Well} 
Because the symmetric potential implies energy eigenfunctions of definite
parity and the ground state eigenfunction is even, $\frac{d\psi(u)}{du}$ must
necessarily be 0 at $x_0 = 0$. 

Rather than worry about the normalization conditions for $\psi(u)$
$\left(\int_{-\infty}^{\infty}\psi^\dag\psi du = 1\right)$, we recognize that
the normalized eigenfunction is simply the unnormalized eigenfunction
multiplied by a constant normalization factor. We can compute the normalization
factor given the unormalized eigenfunction, so we make the assumption that
$\psi = 1.0$ at $x_0 = 0$, recognizing that we are working with the unnormalized
eigenfunction. This does not affect our results for $\epsilon$ because
substitution of $\psi(u)$ times a normalization constant into
\eqref{FiniteBoxSchrodinger} shows that the normalization constants cancel. 

Our simulation will only find eigenfunctions which are real functions. This is
not a problem because we can choose the phase of solutions to the TISE
\eqref{FiniteBoxSchrodinger} to make $\psi(u) \in \mathbb{R}$ (see assignment
specifications 3).

Using these initial conditions, the "shooting" method to determine $\epsilon$ yields:

\begin{figure}[H]
\centering
\sageplot[width=300px, ]{plot_finite_well(shooting_solver_1d_finite(1, 0, 1.5, .01, 0.0001), n=1, title="Guess $\epsilon=0.01$")}
\end{figure}
\vspace{-.2in}
\begin{figure}[H]
\centering
\sageplot[width=300px]{plot_finite_well(shooting_solver_1d_finite(1, 0, 1.5, .09, 0.0001), n=1, title="Guess $\epsilon=0.09$")}
\end{figure}
\vspace{-.2in}
\begin{figure}[H]
\centering
\sageplot[width=300px]{plot_finite_well(shooting_solver_1d_finite(1, 0, 1.5, .095, 0.0001), n=1, title="Guess $\epsilon=0.095$")}
\end{figure}
\vspace{-.2in}
\begin{figure}[H]
\centering
\sageplot[width=300px]{plot_finite_well(shooting_solver_1d_finite(1, 0, 1.5, .0975, 0.0001), n=1, title="Guess $\epsilon=0.0975$")}
\end{figure}
\vspace{-.2in}
\begin{figure}[H]
\centering
\sageplot[width=300px]{plot_finite_well(shooting_solver_1d_finite(1, 0, 1.5, .098, 0.0001), n=1, title="Guess $\epsilon=0.098$")}
\end{figure}
\vspace{-.2in}
\begin{figure}[H]
\centering
\sageplot[width=300px]{plot_finite_well(shooting_solver_1d_finite(1, 0, 1.5, .097993, 0.0001), n=1, title="Ground Energy Eigenfunction, $\epsilon=0.097993$")}
\end{figure}

A valid solution is found at $\epsilon = 0.097993$, implying that the 
ground state energy eigenfunction has an eigenvalue of $0.097993V_0$, where
$V_0$ is the height of the finite well potential. We see that there is a single
node in this wavefunction, confirming that this eigenstate is the ground state.

This solution agrees quite well with the analytic solution $\epsilon = 0.0980$
(ER Appendix H).

\subsection{Excited States of Finite Quantum Well}

Bound states will exist so long as $E < V_0 \implies \epsilon < 1$ so we can
search for additional energy eigenstates by guessing $\epsilon \in [0.0980,
1]$. The first excited state will have a node at the origin ($\psi(0) = 0$).
The linearity property of the Schr\"{o}dinger equation
\eqref{FiniteBoxSchrodinger} allows us to choose any arbitrary value for
$\frac{d\psi}{du}$ (we will use $\frac{d\psi}{du}=1$).

Running the simulation using these initial conditions (multiple trials omitted)
, we find that the first excited energy eigenfunction $\psi_2$ has an energy
eigenvalue $\epsilon = 0.382605 \implies E = 0.382605V_0$:

\begin{figure}[H]
\centering
\sageplot[width=300px]{plot_finite_well(shooting_solver_1d_finite(0, 1, 2, .382605, 0.0001), n=2, title="1st Excited Energy Eigenfunction, $\epsilon=0.382605$")}
\end{figure}

For $\psi_3$, $\epsilon = 0.80767 \implies E = 0.80767V_0$:

\begin{figure}[H]
\centering
\sageplot[width=300px]{plot_finite_well(shooting_solver_1d_finite(1, 0, 2, .80767, 0.0001), n=3, title="2nd Excited Energy Eigenfunction, $\epsilon=0.80767$")}
\end{figure}

\subsubsection{Other Numerical Methods} 
The shooting method provides a convenient way to numerically solve the
finite-well problem by converting the boundary value problem to an initial
value problem with asymptotic boundary constraints.

Another possible numerical method for solving the finite-well problem could
utilize finite difference methods. To accompany asymptotic boundary conditions, 
we choose the boundary conditions at a large value for $u$. Treating $\epsilon$
as an unknown, the multiplication of $\epsilon$ with $\psi(u)$ in equation
\eqref{FiniteBoxSchrodinger} will result in a system of non-linear equations.
This system of non-linear equations can then be solved using Newton's method or
any other numerical root-finding method, yielding both $\epsilon$ as well as
the mappings for the eigenfunction $\psi(u)$ discretized to the same resolution
as the step size of the finite difference schema.

In addition, we can also improve on the method for refining our guess for $\epsilon$. 
Rather than manually update our guess, a binary search procedure could be used to refine
subsequent guesses for $\epsilon$ until a certain error tolerance is reached.

\subsection{"Scattering" States of Finite Quantum Well}\label{sec:ScatterStates}

States where ($E > V_0 \implies \epsilon > 1$) are where the finite quantum
well is no longer able to bind the particle. For these unbound "scattering" states,
the allowed energy levels are continuous. One example of an acceptable unbound solution to
equation \eqref{FiniteBoxSchrodinger} is when $\epsilon = 3$:

\begin{figure}[H]
\centering
\sageplot[width=300px]{plot_finite_well(shooting_solver_1d_finite(-1, 0, 5, 3, 0.0001), n="unbound", title="Unbound Energy State, $\epsilon=3$")}
\end{figure}

In "scattering" states, the particle takes on a wavefunction similar to that of
a 1D free particle. This makes sense as the particle is not bound by the
potential so we expect behavior similar to that of an unbound particle.

In fact, any $\epsilon > 1$ is a valid "scattering" state where the particle
exhibits a behavior similar to that of a free particle. As $\epsilon$ is
increased, so does the frequency of $\psi$. In the limit $E \rightarrow
\infty$, the probability density approaches the clasically expected uniform
probability density for a free particle. 

\subsubsection{Step size error}\label{StepSizeError}
This result, however, is quite sensitive to the step size used for numerical
integration. Forward Euler is a first order solver which can be numerically
unstable and introduce additional energy. For example, the results from
section \ref{sec:ScatterStates} no longer result in sinusoidal asymptotic
behavior when $\Delta U$ is changed from $0.0001$ to $0.01$.  Rather, the
Forward Euler approximation adds enough energy to result in infinite blowup:

\begin{figure}[H]
\vspace{-.1in}
\centering
\sageplot[width=300px]{plot_finite_well(shooting_solver_1d_finite(-1, 0, 5, 3, 0.01), n="unbound", title="Energy gain due to step size=0.01, $\epsilon=3$")}
\end{figure}

To address the numerical stability issues of Forward-Euler as well as increase
rate of convergance of error to $0$ as $\Delta U \rightarrow 0$, implicit
and symplectic solvers of higher order (eg. Runge-Katta) can be used to carry out numerical
integration.

\section{Infinite Quantum Well}\label{sec:InfWell}

The simulation code can be easily adjusted for different potentials. For
example, the infinite quantum well can be simulated by modifying the definition
of the potential:

\begin{sageblock}
# 999 to approximate infinite potential
V_potential = lambda u: 999 if (abs(u) > 1/2) else 0
\end{sageblock}

Next, by interpreting $V_0 = 1$ eV and using the boundary condition
$\psi(\text{\(\lvert u\rvert\)}=\frac{1}{2}) = 0$, the energy eigenstates can be found.

Since we have an analytical solution for an infinite square well (ER Eq. 6-79,
Eq. 6-80):

\begin{equation}
    \psi_n(x) = 
    \begin{cases}
        B_n \cos k_n x &\text{where } k_n = \frac{n\pi}{a},\; n = 1,3,5,\ldots\\
        A_n \sin k_n x &\text{where } k_n = \frac{n\pi}{a},\; n = 2,4,6,\ldots\\
    \end{cases}
\end{equation}

We can visualize how well our simulation approximates analytical results by modifying our
plotting function to also plot the analytical solutions. The constant factors $A_n$ and $B_n$
are chosen to be consistent with initial conditions:

\begin{align}
    \psi(0) &= B_n \\
    \frac{d\psi}{du}\bigg|_{u=0} &= \frac{d}{dx}A_n \sin k_n a\cdot0 \nonumber \\
                        &= A_n k_n \cos k_n a\cdot0 \nonumber \\
                        &= A_n k_n
\end{align}

Solving these using the initial conditions $\psi(0)=1$ and
$\frac{d\psi}{du}\big|_{u=0}=1$ yields $A_n = \frac{1}{k_n}$ and $B_n = 1$,

We use these results to define a \texttt{plot\_inf\_well} method which
takes a wave number \texttt{n}, a \texttt{data} list of simulation results,
and a \texttt{title} string.  This function will first plot the analytical
solution for $\psi_n(u)$ in an $\infty$-square well potential and subsequently
make a chained call to \texttt{plot\_finite\_well} to plot the simulation
results with dashed well boundaries.

\begin{sageblock}
from numpy import arange
def plot_inf_well(n, uf, data, title):
    plt.clf() # clear plot
    u = arange(-float(uf),float(uf),2*float(uf)/100) # generate timespan array
    # determine the proper analytic solution to overlay
    if n % 2 != 0: # eigenfunction is even
        psi = [1 * cos(n*i*pi) for i in u]
    else: # eigenfunction is odd
        psi = [(1 / (n * pi)) * sin(n*i*pi) for i in u]
    plt.plot(u, psi, label="Analytical Solution", marker='+', linestyle='None', color='Green')
    plt.save = plt.savefig # hack for proper SageTeX behavior
    return plot_finite_well(data, n, title, chained=True) # chain to plot_finite_well
\end{sageblock}

This results in (multiple trials omitted):

\begin{figure}[H]
\centering
\sageplot[width=300px]{plot_inf_well(1, 0.5, shooting_solver_1d_finite(1, 0, 0.5, .1542, 0.0001), title="Ground Energy Eigenfunction, $\infty$-quantum well, $\epsilon=0.1542$")}
\end{figure}

\begin{figure}[H]
\centering
\sageplot[width=300px]{plot_inf_well(2, 0.5, shooting_solver_1d_finite(0, 1, 0.5, 0.6179, 0.0001), title="First Exctited Energy Eigenfunction, $\infty$-quantum well, $\epsilon=0.6179$")
}
\end{figure}

We see that our simulated eigenfunctions follow the analytical solutions very
closely. The results indicate that the ground and first excited energy
eigenstates for this $\infty$-quantum well have energy eigenvalues of 0.1542 eV
and 0.6179 eV.  Compared to the energy eigenvalues obtained analytically (ER Eq.
6-81):

\begin{align}
    m &= \frac{\beta\hbar^2}{2a^2V_0} \:(\text{from } \eqref{DimensionlessSubstitutions})\nonumber \\
    E_n &= \frac{\pi^2 \hbar^2 n^2}{2ma^2} = \frac{\pi^2n^2V_0}{\beta} \nonumber \\
    E_1 &= \frac{\pi^2}{64} \approx 0.1542 \\
    E_2 &= \frac{4\pi^2}{64} \approx 0.6168 
\end{align}

we see that our simulation yields an answer accurate to one thousandth of an eV.

\section{Quantum Harmonic Oscillator}
The numerical methods developed so far can be extended to other non-trivial
potentials. For a quantum harmonic oscillator, the potential is defined by
$V(x) = \frac{1}{2}Cx^2$ and has a natural frequency of $\omega =
\sqrt{\frac{C}{m}}$.  We can reformulate the time-independent Schr\"{o}dinger
equation for the quantum harmonic oscillator into a dimensionless form (P24 -
PS1 Problem 3):

\begin{equation}\label{eq:qho-tise}
    \frac{d^2\psi}{du^2} + (\epsilon - u^2)\psi = 0
\end{equation}

where:

\begin{equation}
    \epsilon = \frac{2E}{\hbar\omega},\;\;
    u = \frac{x}{x_0},\;\;
    x_0 = \sqrt{\frac{\hbar}{m\omega}}\;\;
\end{equation}

While section \ref{sec:InfWell} illustrated that the potential can be modified by simply changing
one line, because we have reparameterized the TISE to a dimensionless quantum harmonic oscillator
case, we will need to define a new \texttt{shooting\_solver\_qho} method which implements the 
reparameterized TISE (\ref{eq:qho-tise}). Additionally, we will define a
modified \texttt{plot\_qho} which does not plot the well boundaries at $u=\pm0.5$.

\begin{sageblock}
hbar = 6.58211928*10**-16 # eV s
V_potential = lambda u: C * (u * x0)**2 / 2
C = 1
m = 1
omega = sqrt(C / m)
x0 = sqrt(hbar / (m * omega))

def shooting_solver_qho(psi0, dpsi0, uf, epsilon, delta_u):
    num_steps = (uf - 0) / delta_u
    data = [(0, psi0, dpsi0)] # initialize data array
    for i in range(num_steps): # perform forward euler
        u_old = data[i][0]
        psi_old = data[i][1]
        dpsi_old = data[i][2]
        u = u_old + delta_u
        # Taylor approximations using (10)
        dpsi = dpsi_old - delta_u * (epsilon - u**2) * psi_old
        psi = psi_old + delta_u * dpsi_old
        data.append((u, psi, dpsi))
    return map(lambda x: (x[0], x[1]), data) # return list of (u, psi) tuples

def plot_qho(data, n, title, chained=False):
    if not chained: plt.clf() # only clear if not part of chained call
    # mirror data across origin, assumes data has a definite parity
    (u, psi0) = data[0]
    if psi0 == 0: # node at 0, odd
        data = map(lambda x: (-x[0], -x[1]), data)[::-1] + data
    else: # definite parity => anti-node at 0, even
        data = map(lambda x: (-x[0], x[1]), data)[::-1] + data

    u, psi = [[x[i] for x in data] for i in (0,1)]
    plt.plot(u, psi, label="Euler Approximation")
    plt.title(title)
    plt.xlabel('$u$')
    plt.ylabel("$\\psi_{%s}(u)$" % n)
    plt.axvline(x=0, color='black')
    plt.grid(True)
    plt.legend()
    plt.save = plt.savefig
    return plt 
\end{sageblock}

Again, we use the symmetry of the potential and only simulate for $u \ge 0$,
appropriately mirroring the results in \texttt{plot\_qho}. Also, the definite
parity of eigenfunctions due to the symmetric potential and the linearity of
eigenstates again allow us to arbitrarily choose initial conditions of
$\psi(0)=1$ for even eigenfunctions and $\frac{d\psi}{du}\big|_{u=0}=1$ for odd
eigenfunctions. We can now repeatedly guess and refine an estimate for
$\epsilon$ which causes $\psi$ to satisfy the asymptotic bound $\psi(u) = 0$ as $u
\rightarrow \infty$ (multiple trials omitted):

\begin{figure}[H]
\centering
\sageplot[width=300px]{plot_qho(shooting_solver_qho(1, 0, 3.9, 1.00005, 0.0001), n=1, title="QHO Ground State, $\epsilon=1.00005$")}
\end{figure}

The first excited state can be found by requiring a node at $u = 0$, which we
do by setting the initial condition $\psi(0)=0$. Since the first excited state 
for a QHO is odd, we use the initial condition $\frac{d\psi}{du}\big|_{u=0}=1$:

\begin{figure}[H]
\centering
\sageplot[width=300px]{plot_qho(shooting_solver_qho(0, 1, 3.9, 3, 0.0001), n=2, title="QHO First Excited, $\epsilon=3.9$")}
\end{figure}

Our results estimate the dimensionless energy parameter
$\epsilon=\frac{2E}{\hbar\omega}$ to be $\epsilon_1=1.00005$ in the ground
state and $\epsilon_2=3.9$ in the first excited state. This agrees with
analytical results (P24 - PS3 - Problem 3c), which require $\epsilon_n = 1, 3,
5, \ldots$

\section{Conclusions}

This report shows that the shooting method coupled with numerical integration
using Forward Euler can produce satisfactory approximations of eigenfunctions
and eigenvalues for the TISE. The code accompanying this report not only solve
the 1D finite well with results very close to analytical solutions, but is
also modular and easily adaptable to different potentials.  This was
illustrated by the relatively small amount of effort required to reconfigure
our program to solve the infinite well potential and quantum harmonic
oscillator potential. In both of these latter examples, numerical techniques
achieved results in close agreement with analytical solutions.

By decreasing the step size used for numerical integration, the error
introduced by Forward Euler should converge towards zero, yielding a flexible
method to numerically integrate open-form differential equations. Coupled with
the plotting routines presented, this set of methods allows us to conveniently
approximate energy eigenvalues and eigenstates for arbitrary 1D potentials and
plot the results.

Nonetheless, our method is far from perfect. We saw in section
\ref{StepSizeError} that Forward Euler violates the conservation of energy and
can result in unstable asymptotic behavior. Decreasing the step size is one
solution to this problem, but doing so also increases the amount of computation
required to carry out the simulation. Other methods of increasing the rate of
error convergence to 0 as $\Delta U \rightarrow 0$ include using a higher-order
implicit/symplectic solver for the numerical integration process. To improve
the precision of $\epsilon$, the repetitive guess-refine feedback loop
necessary for the shooting method can be itself made a callable bisection search
over the $\epsilon$ parameter with some termination condition.

\end{document}
